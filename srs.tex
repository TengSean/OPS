%Copyright 2014 Jean-Philippe Eisenbarth
%This program is free software: you can 
%redistribute it and/or modify it under the terms of the GNU General Public 
%License as published by the Free Software Foundation, either version 3 of the 
%License, or (at your option) any later version.
%This program is distributed in the hope that it will be useful,but WITHOUT ANY 
%WARRANTY; without even the implied warranty of MERCHANTABILITY or FITNESS FOR A 
%PARTICULAR PURPOSE. See the GNU General Public License for more details.
%You should have received a copy of the GNU General Public License along with 
%this program.  If not, see <http://www.gnu.org/licenses/>.

%Based on the code of Yiannis Lazarides
%http://tex.stackexchange.com/questions/42602/software-requirements-specification-with-latex
%http://tex.stackexchange.com/users/963/yiannis-lazarides
%Also based on the template of Karl E. Wiegers
%http://www.se.rit.edu/~emad/teaching/slides/srs_template_sep14.pdf
%http://karlwiegers.com
\documentclass{scrreprt}
\usepackage{listings}
\usepackage{underscore}
\usepackage[bookmarks=true]{hyperref}
\usepackage[utf8]{inputenc}
\usepackage[english]{babel}
\usepackage{CJKutf8}
\CJKencfamily{UTF8}{bkai}
\hypersetup{
    bookmarks=false,    % show bookmarks bar?
    pdftitle={Software Requirement Specification},    % title
    pdfauthor={Jean-Philippe Eisenbarth},                     % author
    pdfsubject={TeX and LaTeX},                        % subject of the document
    pdfkeywords={TeX, LaTeX, graphics, images}, % list of keywords
    colorlinks=true,       % false: boxed links; true: colored links
    linkcolor=blue,       % color of internal links
    citecolor=black,       % color of links to bibliography
    filecolor=black,        % color of file links
    urlcolor=purple,        % color of external links
    linktoc=page            % only page is linked
}%
\def\myversion{2.0 }
\date{}
%\title
\usepackage{hyperref}
\begin{document}
\begin{CJK}{UTF8}{}
\begin{flushright}
    \rule{16cm}{5pt}\vskip1cm
    \begin{bfseries}
        \Huge{SOFTWARE REQUIREMENTS\\ SPECIFICATION}\\
        \vspace{1.8cm}
        for\\
        \vspace{1.8cm}
        開放平台期末專題-標題\\
        \vspace{1.8cm}
        \LARGE{Version \myversion approved}\\
        Prepared by 1041562 吳宗晉\\
        1041562 黃騰昇\\
        1041562 施紹唐\\
        \vspace{1.8cm}
        第十組\\
        \vspace{1.8cm}
        \today\\
    \end{bfseries}
\end{flushright}

\tableofcontents

\chapter{導論}

\section{目的}
以往學習影像辨識時都以辨識區分人臉為主,因此希望將相關知識套用在動物的種類區分上,為生態保育貢獻心力

\section{目標族群與說明}
對於以需要以大規模的影像標籤動物的人或組織。

\section{專案範疇}
必要範疇: 該專案需要在kaggle上正確運行訓練出模型,並能在輸入圖片之後辨識之,且準確率達水準之上。
充要範疇: 

\chapter{整體說明}
我們挑選的主題是Categorize animals in the wild,一個在kaggle上面舉辦的比賽,利用kaggle提供的資料集來訓練出一個模型,讓我們可以藉由這個模型來為輸入影像分類,模型可以區分22種動物,希望可以藉由訓練出模型來監測觀察不同地區生物生態以及遷移情況,讓研究人員可以更輕易的掌握生態變遷。

\section{方向}
該系統旨在

\section{產品功能}
對於初次運行的終端,系統會進行初步訓練,
而後將欲的圖片存儲在系統資料夾下,系統會對圖片辨識

\section{使用者類型和特性}
A.野外生態的觀察者
B.動物保育人員或者相關組織


\section{運行環境}
使用kaggle終端和jupyter notebook

\section{設計與實作上困難}
主要是硬體方面的困難,包含
1.gpu加速問題導致執行時間過長
2.內存不足的問題
3.基於transferlearning運行densenet的時候,沒辦法套用imagenet的權重

\section{假設性和依賴性}


\chapter{外部界面需求}

\section{使用者界面}
使用kaggle所提供之終端機介面進行訓練和辨識
\section{硬體界面}

\section{軟體界面}

\chapter{系統特色}
\section{描述與優先權}
\section{回傳序列}
\section{功能需求}
%\begin{center}
%\begin{tabular}{|l|l|l|}
%	\hline
%	ID & 功能名稱 & 描述 
%	\hline
%\end{tabular}
%\end{center}

\chapter{其他非功能性需求}

\section{運行需求}
需要使用jupyter notebook
\section{安全性需求}

\section{隱私性需求}


\end{CJK}
\end{document}
