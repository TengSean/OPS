%Copyright 2014 Jean-Philippe Eisenbarth
%This program is free software: you can 
%redistribute it and/or modify it under the terms of the GNU General Public 
%License as published by the Free Software Foundation, either version 3 of the 
%License, or (at your option) any later version.
%This program is distributed in the hope that it will be useful,but WITHOUT ANY 
%WARRANTY; without even the implied warranty of MERCHANTABILITY or FITNESS FOR A 
%PARTICULAR PURPOSE. See the GNU General Public License for more details.
%You should have received a copy of the GNU General Public License along with 
%this program.  If not, see <http://www.gnu.org/licenses/>.

%Based on the code of Yiannis Lazarides
%http://tex.stackexchange.com/questions/42602/software-requirements-specification-with-latex
%http://tex.stackexchange.com/users/963/yiannis-lazarides
%Also based on the template of Karl E. Wiegers
%http://www.se.rit.edu/~emad/teaching/slides/srs_template_sep14.pdf
%http://karlwiegers.com
\documentclass{scrreprt}
\usepackage{listings}
\usepackage{fontspec}
\usepackage{xeCJK}
\XeTeXlinebreaklocale "zh"
\XeTeXlinebreakskip = 0pt plus 1pt
\setCJKmainfont{標楷體} 
\usepackage{underscore}
\usepackage{graphicx}
\usepackage{titletoc}
\usepackage[bookmarks=true]{hyperref}
%\usepackage[utf8]{inputenc}
%\usepackage[english]{babel}
%\usepackage{CJKutf8}
%\CJKencfamily{UTF8}{bkai}
\hypersetup{
    bookmarks=false,    % show bookmarks bar?
    pdftitle={Software Requirement Specification},    % title
    pdfauthor={施紹唐},                     % author
    pdfsubject={TeX and LaTeX},                        % subject of the document
    pdfkeywords={TeX, LaTeX, graphics, images}, % list of keywords
    colorlinks=true,       % false: boxed links; true: colored links
    linkcolor=blue,       % color of internal links
    citecolor=black,       % color of links to bibliography
    filecolor=black,        % color of file links
    urlcolor=purple,        % color of external links
    linktoc=page            % only page is linked
}%
\def\myversion{2.0 }
\date{}
%\title
\usepackage{hyperref}
\begin{document}
%\begin{CJK}{UTF8}{}
\begin{flushright}
    \rule{14cm}{5pt}\vskip1cm
    \begin{bfseries}
        \Huge{SOFTWARE REQUIREMENTS\\ SPECIFICATION}\\
        \vspace{1.8cm}
        for\\
        \vspace{1.8cm}
        開放平台期末專題\\
        \vspace{1.8cm}
        \LARGE{Version \myversion approved}\\
		1041562 吳宗晉\\
        1041562 黃騰陞\\
        Prepared by 1043310 施紹唐\\
        \vspace{1.8cm}
        第十組\\
        \vspace{1.8cm}
        \today\\
    \end{bfseries}
\end{flushright}

\tableofcontents

\chapter{導論}
\section{目的}
以往學習影像辨識時都以辨識區分人臉為主,因此希望將相關知識套用在動物的種類區分上,為生態保育貢獻一份心力。

\section{目標族群與說明}
對於以需要以大規模的影像標籤動物的人或組織。國外已有人開發出相關的app,只要一人將所見的動物拍照並登陸地點,即可協助統計該地區內的物種資料。

\section{專案範疇}
\begin{itemize}
\item 必要範疇: 該專案需要在kaggle上正確運行訓練出模型,並能在輸入測試圖片之後辨識分類之,且準確率需達水準之上。
\item 充要範疇: 擁有基本的人機介面。
\end{itemize}


\chapter{整體說明}
\section{方向}
我們挑選的主題是Categorize animals in the wild,一個在kaggle上面舉辦的比賽,利用kaggle提供的資料集來訓練出一個模型,讓我們可以藉由這個模型來為輸入影像分類,希望可以藉由訓練出模型來監測觀察不同地區生物生態以及遷移情況,讓研究人員可以更輕易的掌握生態變遷。

\section{產品功能}
可以針對輸入的圖像將其分類成23種物種中的一種。若是圖像與這23個物種都不相似,則會將其歸類為其他。

\section{使用者類型和特性}
\begin{itemize}
\item 野外生態的任意觀察者
\item 動物保育人員或者相關組織
\end{itemize}

\section{運行環境}
training使用kaggle核心,因為kaggle提供gpu加速。辨識使用jupyter notebook。環境為windows系統。

\section{設計與實作上困難}
主要是硬體方面的困難,包含:
\begin{itemize}
	\item gpu加速問題導致執行時間過長。
	\item 內存不足的問題。
	\item 基於transferlearning運行densenet的時候,沒辦法套用imagenet的權重。
\end{itemize}


\chapter{外部界面需求}
\section{使用者界面}
直接運行jupyter notebook以pyplot獲取辨識曲線。
\section{硬體界面}
無。
\section{軟體界面}
無。
\chapter{系統特色}
\section{描述與優先權}
必須優先解決內存問題,否則model將無法被產生。
\section{回傳序列}
Stimulus: 請求解決訓練時系統內存不足的問題。 \\
Response: 將讀入圖片時將圖片由720*1024縮放成32*32大小。 \\ 
Stimulus: 請求更換權重,因為imagenet的權重無法被套用在densenet121上。 \\
Response: 已將權重改為網路上與imagenet相似的權重,在kaggle上也有許多人採用。 \\
\section{功能需求}
設計流程
\begin{center}
	\begin{tabular}[t]{|l|l|p{10cm}|}
		\hline
		ID & 功能名稱 & 描述 \\
		\hline
		1 & 資料集的讀取 & 為了訓練模型,需要讀取訓練用圖片與驗證的資料集(格式為csv)。 \\
		\hline
		2 & 前處理 & 主要是對影像resize和normalize,把值量化到[0-1]。 \\
		\hline
		3 & 建立模型 & 運用densenet121建構出模型。 \\
		\hline
		4 & 模型優化 & 運用optimizer優化loss function使誤差值達到最小。 \\
		\hline
	\end{tabular}
\end{center}
\chapter{其非功能性需求}
\section{運行需求}
訓練model的時間約為5個小時。辨識時間則視圖片多寡而定。

%\end{CJK}
\end{document}
