%Copyright 2014 Jean-Philippe Eisenbarth
%This program is free software: you can 
%redistribute it and/or modify it under the terms of the GNU General Public 
%License as published by the Free Software Foundation, either version 3 of the 
%License, or (at your option) any later version.
%This program is distributed in the hope that it will be useful,but WITHOUT ANY 
%WARRANTY; without even the implied warranty of MERCHANTABILITY or FITNESS FOR A 
%PARTICULAR PURPOSE. See the GNU General Public License for more details.
%You should have received a copy of the GNU General Public License along with 
%this program.  If not, see <http://www.gnu.org/licenses/>.

%Based on the code of Yiannis Lazarides
%http://tex.stackexchange.com/questions/42602/software-requirements-specification-with-latex
%http://tex.stackexchange.com/users/963/yiannis-lazarides
%Also based on the template of Karl E. Wiegers
%http://www.se.rit.edu/~emad/teaching/slides/srs_template_sep14.pdf
%http://karlwiegers.com
\documentclass{scrreprt}
\usepackage{listings}
\usepackage{underscore}
\usepackage[bookmarks=true]{hyperref}
\usepackage[utf8]{inputenc}
\usepackage[english]{babel}
\usepackage{CJKutf8}
\CJKencfamily{UTF8}{bkai}
\hypersetup{
    bookmarks=false,    % show bookmarks bar?
    pdftitle={Software Requirement Specification},    % title
    pdfauthor={Jean-Philippe Eisenbarth},                     % author
    pdfsubject={TeX and LaTeX},                        % subject of the document
    pdfkeywords={TeX, LaTeX, graphics, images}, % list of keywords
    colorlinks=true,       % false: boxed links; true: colored links
    linkcolor=blue,       % color of internal links
    citecolor=black,       % color of links to bibliography
    filecolor=black,        % color of file links
    urlcolor=purple,        % color of external links
    linktoc=page            % only page is linked
}%
\def\myversion{2.0 }
\date{}
%\title
\usepackage{hyperref}
\begin{document}
\begin{CJK}{UTF8}{}
\begin{flushright}
    \rule{16cm}{5pt}\vskip1cm
    \begin{bfseries}
        \Huge{SOFTWARE REQUIREMENTS\\ SPECIFICATION}\\
        \vspace{1.8cm}
        for\\
        \vspace{1.8cm}
        開放平台期末專題-SRS\\
        \vspace{1.8cm}
        \LARGE{Version \myversion approved}\\
        Prepared by 1041562 吳宗晉\\
        1041562 黃騰昇\\
        1041562 施紹唐\\
        \vspace{1.8cm}
        第十組\\
        \vspace{1.8cm}
        \today\\
    \end{bfseries}
\end{flushright}

\tableofcontents


\chapter*{Revision History}

\begin{center}
    \begin{tabular}{|c|c|c|c|}
        \hline
	    Name & Date & Reason For Changes & Version\\
        \hline
	    21 & 22 & 23 & 24\\
        \hline
	    31 & 32 & 33 & 34\\
        \hline
    \end{tabular}
\end{center}

\chapter{導論}

\section{目的}


\section{適用對象與說明}
我們挑選的主題是Categorize animals in the wild,一個在kaggle上面舉辦的比賽,利用kaggle提供的資料集來訓練出一個模型,讓我們可以藉由這個模型來為輸入影像分類,模型可以區分22種動物,希望可以藉由訓練出模型來監測觀察不同地區生物生態以及遷移情況,讓研究人員可以更輕易的掌握生態變遷。

\section{計畫範圍}


\chapter{整體說明}
\section{作品方向}
\section{作品功能}
\section{使用者類型和特性}
\section{運行環境}
\section{設計與實作上困難}
\section{假設性和依賴性}

\chapter{外部界面需求}
\section{使用者界面}
\section{硬體界面}
\section{軟體界面}

\chapter{系統特色}
\section{描述與優先權}
\section{回傳序列}
\section{功能需求}

\chapter{其他非功能性需求}
\section{運行需求}
\section{安全性需求}
\section{隱私性需求}


\section{模型的輸入資料}
%dataset csv

\section{模型的輸出資料}


\section{模型架構}


\section{如何取出模型}


\section{Loss function選擇與採用}


\section{optimizer選擇與採用}






\chapter{Dataset}

\section{資料集大小}


\section{資料集來源}


\section{訓練樣本數量}


\section{驗證樣本數量}


\section{測試樣本數量}


\chapter{Experimental Evaluation}
$<$This template illustrates organizing the functional requirements for the 
product by system features, the major services provided by the product. You may 
prefer to organize this section by use case, mode of operation, user class, 
object class, functional hierarchy, or combinations of these, whatever makes the 
most logical sense for your product.$>$

\section{硬體環境}
$<$Don’t really say “System Feature 1.” State the feature name in just a few 
words.$>$

\section{訓練次數}
$<$Provide a short description of the feature and indicate whether it is of 
High, Medium, or Low priority. You could also include specific priority 
component ratings, such as benefit, penalty, cost, and risk (each rated on a 
relative scale from a low of 1 to a high of 9).$>$

\subsection{}
$<$List the sequences of user actions and system responses that stimulate the 
behavior defined for this feature. These will correspond to the dialog elements 
associated with use cases.$>$

\section{Functional Requirements}
$<$Itemize the detailed functional requirements associated with this feature.  
These are the software capabilities that must be present in order for the user 
to carry out the services provided by the feature, or to execute the use case.  
Include how the product should respond to anticipated error conditions or 
invalid inputs. Requirements should be concise, complete, unambiguous, 
verifiable, and necessary. Use “TBD” as a placeholder to indicate when necessary 
information is not yet available.$>$

$<$Each requirement should be uniquely identified with a sequence number or a 
meaningful tag of some kind.$>$

REQ-1:	REQ-2:

\section{System Feature 2 (and so on)}


\chapter{Demo}

\section{Performance Requirements}
$<$If there are performance requirements for the product under various 
circumstances, state them here and explain their rationale, to help the 
developers understand the intent and make suitable design choices. Specify the 
timing relationships for real time systems. Make such requirements as specific 
as possible. You may need to state performance requirements for individual 
functional requirements or features.$>$

\section{Safety Requirements}
$<$Specify those requirements that are concerned with possible loss, damage, or 
harm that could result from the use of the product. Define any safeguards or 
actions that must be taken, as well as actions that must be prevented. Refer to 
any external policies or regulations that state safety issues that affect the 
product’s design or use. Define any safety certifications that must be 
satisfied.$>$

\section{Security Requirements}
$<$Specify any requirements regarding security or privacy issues surrounding use 
of the product or protection of the data used or created by the product. Define 
any user identity authentication requirements. Refer to any external policies or 
regulations containing security issues that affect the product. Define any 
security or privacy certifications that must be satisfied.$>$

\section{Software Quality Attributes}
$<$Specify any additional quality characteristics for the product that will be 
important to either the customers or the developers. Some to consider are: 
adaptability, availability, correctness, flexibility, interoperability, 
maintainability, portability, reliability, reusability, robustness, testability, 
and usability. Write these to be specific, quantitative, and verifiable when 
possible. At the least, clarify the relative preferences for various attributes, 
such as ease of use over ease of learning.$>$

\section{Business Rules}
$<$List any operating principles about the product, such as which individuals or 
roles can perform which functions under specific circumstances. These are not 
functional requirements in themselves, but they may imply certain functional 
requirements to enforce the rules.$>$


\chapter{Other Requirements}
$<$Define any other requirements not covered elsewhere in the SRS. This might 
include database requirements, internationalization requirements, legal 
requirements, reuse objectives for the project, and so on. Add any new sections 
that are pertinent to the project.$>$

\section{Appendix A: Glossary}
%see https://en.wikibooks.org/wiki/LaTeX/Glossary
$<$Define all the terms necessary to properly interpret the SRS, including 
acronyms and abbreviations. You may wish to build a separate glossary that spans 
multiple projects or the entire organization, and just include terms specific to 
a single project in each SRS.$>$

\section{Appendix B: Analysis Models}
$<$Optionally, include any pertinent analysis models, such as data flow 
diagrams, class diagrams, state-transition diagrams, or entity-relationship 
diagrams.$>$

\section{Appendix C: To Be Determined List}
$<$Collect a numbered list of the TBD (to be determined) references that remain 
in the SRS so they can be tracked to closure.$>$
\end{CJK}
\end{document}
